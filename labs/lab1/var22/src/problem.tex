\section{Условие}

{\bfseries Цель работы:} \\
Приобретение практических навыков в:
\begin{itemize}
\item Управлении процессами в ОС
\item Обеспечение обмена данных между процессами посредством каналов
\end{itemize}

{\bfseries Задание:} \\
Составить и отладить программу на языке Си, осуществляющую работу с процессами и
взаимодействие между ними в одной из двух операционных систем. В результате работы
программа (основной процесс) должен создать для решение задачи один или несколько
дочерних процессов. Взаимодействие между процессами осуществляется через системные
сигналы/события и/или каналы (pipe).
Необходимо обрабатывать системные ошибки, которые могут возникнуть в результате работы.

\begin{figure}[H]
    \centering
    \includegraphics[width=0.75\textwidth]{schema.png}
    \caption{Схема работы процессов.}
    \label{fig:schema}
\end{figure}

{\bfseries Вариант:} 22 \\
Родительский процесс создает два дочерних процесса. Первой строкой пользователь в консоль
родительского процесса вводит имя файла, которое будет использовано для открытия File с таким
именем на запись для child1. Аналогично для второй строки и процесса child2. Родительский и
дочерний процесс должны быть представлены разными программами.
Родительский процесс принимает от пользователя строки произвольной длины и пересылает их в
pipe1 или в pipe2 в зависимости от правила фильтрации. Процесс child1 и child2 производят работу
над строками. Процессы пишут результаты своей работы в стандартный вывод. 
Правило фильтрации: с вероятностью 80\% строки отправляются в pipe1, иначе в pipe2.
Дочерние процессы инвертируют строки.