\section{Результаты}

\begin{figure}[H]
    \centering
    \includegraphics[width=0.75\textwidth]{src/image.png}
    \caption{График зависимости времени от количества потоков.}
    \label{fig:graph}
\end{figure}

Результатом работы программы является время (мс) за которое алгоритм выполняет кластеризацию и количество итераций, которое понадобилось.
При входных данных 10 000 точек и 500 класстеров получены следующие данные:
\begin{itemize}
\item 1 поток --- 28 итераций, 5771 мс -> ~206 мс на итерацию. (BASE)
\item 2 потока --- 27 итераций, 2764 мс -> ~102 мс на итерацию. (x2.0)
\item 3 потока --- 28 итераций, 2064 мс -> ~73 мс на итерацию. (x2.8)
\item 4 потока --- 22 итерации, 1358 мс -> ~66 мс на итерацию. (x3.1)
\item 5 потоков --- 53 итерации, 2766 мс -> ~52 мс на итерацию. (x4.0)
\item 6 потоков --- 31 итерация, 1320 мс -> ~43 мс на итерацию. (x4.8)
\item 7 потоков --- 25 итераций, 978 мс -> ~39 мс на итерацию. (x5.3)
\item 8 потоков --- 20 итераций, 698 мс -> 35 мс на итерацию. (x5.9)
\item 9 потоков --- 28 итераций, 1145 мс -> 41 мс на итерацию. (x5.0)
\end{itemize}
