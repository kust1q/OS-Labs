\section{Метод решения}

Алгоритм решения задачи:

\begin{enumerate}
\item Пользователь запускает программу с целочисленным ключом запуска, означающем максимальное количество потоков которое может использовать программа.

\item Пользователь вводит в консоль количество кластеров - n.

\item Генерируется массив из 10 000 точек на плоскости Oxy и добавляются в общий массив.

\item Первыми n точчками инициализируются центройды кастеров. 

\item Каждый поток проверяет к какому кластеру точка в "своей" части массива точек и ближе и перемещает ее туда.

\item После того, как все потоки отработали, главный поток пересчитывает центройды.

\item Если центройды изменили свое положение, то потоки снова начинают перемещать точки.

\item Алгоритм завершается если положение центройдов не изменилось или количество итераций превысило 1000.

\end{enumerate}

\vspace{1\baselineskip}

Архитектура программы:

\dirtree{%
.1 lab2/.
.2 build/.
.2 include/.
.3 thread.h.
.3 exceptions.h.
.2 src/.
.3 thread.cpp.
.2 main.cpp.
}

\section{Описание программы}

\texttt{main.cpp} --- реализация алгоритма кластеризации k средних. \\
Основные функции:
\begin{itemize}
\item \texttt{void updateCentroids();} --- вспомогательная функция изменения центройдов кластеров.
\item \texttt{void* updateClusters(void* threadData);} --- вспомагательная функция "перемещения" точек в кластеры.
\item \texttt{kMeans();} --- основная функция реализации алгоритма кластеризации методом k средних.
\end{itemize}

\texttt{exceptions.h} --- объявление необходимых исключений.
\begin{itemize}
\item \texttt{CreateThreadException} --- ошибка создания потока.
\item \texttt{WaitThreadException} --- ошибка ожидания завершения потока.
\end{itemize}

\texttt{include/thread.h} --- объявления методов класса thread. \\ 
\texttt{src/thread.cpp} --- реализация.\\

Основные функции:
\begin{itemize}
\item \texttt{void Run(void* threadData);} --- создание и запуск. Используется системный вызов \texttt{pthread\_create(...)}.
\item \texttt{void Join();} --- ожидание завершения потока. \\
Используется системный вызов \texttt{pthread\_join(...)}.
\end{itemize}

