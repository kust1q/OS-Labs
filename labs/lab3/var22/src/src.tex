\section{Метод решения}

Алгоритм решения задачи:

\begin{enumerate}
\item Пользователь в консоль родительского процесса вводит имена файлов, которые будут использованы для открытия (создания) файлов с таким именем для child1 и child2 соответственно.

\item Создается объект класса Parent, так называемый ``родитель''.

\item Родительский процесс создает два объекта разделяемой памяти для передачи данных между родительским и дочерними процессами.

\item Затем делает fork для создания первого дочернего процесса. Пытается запустить бинарный файл для дочернего процесса и передать аргументы - имя разделяемой памяти и имя файла для записи. Если не получилось — процесс завершается. Аналогично для второго дочернего процесса.

\item Дочерний процесс открывает свою область разделяемой памяти. И пытается открыть (создать) указанный файл в специальной директории. Если не получилось — процесс завершается.

\item Родительский процесс проверяет ``жизнидеятельность'' процессов, если любой из них уже завершен, то программа прекращает работу со статусом ошибки.

\item Родительский процесс считывает пользовательский ввод и отправляет строки данных в соответствующие области разделяемой памяти, затем отправляет сигнал одному из дочерних процессов, уведомляя о наличии данных.

\item Дочерний процесс ожидает сигнал, читает строку из разделяемой памяти, инвертирует её и записывает в соответствующий файл.

\item По команде ``exit'' или ``quit'' родитель останавливает дочерние процессы и программа завершается.
\end{enumerate}

\vspace{10\baselineskip}

Архитектура программы:

\dirtree{%
.1 lab1/. 
.2 bin/. 
.3 child.cpp.
.2 build/.
.2 include/.
.3 child.h.
.3 exceptions.h.
.3 os.h.
.3 parent.h.
.2 src/.
.3 child.cpp.
.3 os.cpp.
.3 parent.cpp.
.2 testfiles/.
.2 main.cpp.
}


\section{Описание программы}

\texttt{main.cpp} --- точка входа в программу, создается объект класса Parent, обрабатываются исключения.\\

\texttt{bin/child.cpp} --- точка входа в программу для дочернего процесса, создается объект класса Child.\\

\texttt{exceptions.h} --- объявление необходимых исключений.
\begin{itemize}
\item \texttt{ChildProcessEndException} --- ошибка внезапного завершения любого из дочерних процессов, программа завершается.
\end{itemize}

\texttt{os.h} --- объявление функций управления процессами и разделяемой памятью ОС. \\
\texttt{src/os.cpp} --- реализация.\\
Основные функции:
\begin{itemize}
\item \texttt{ProcessHandle CreateProcess(const ProcessParams\& params);} --- создание дочернего процесса. Используется системный вызов \texttt{fork()} и \texttt{execl()}.
\item \texttt{void TerminateProcess(ProcessHandle handle);} --- завершение процесса по handle и очистка ресурсов. Используется системный вызов \texttt{kill()} и \texttt{waitpid()}.
\item \texttt{bool IsAliveProcess(ProcessHandle handle);} --- проверка ``жизнидеятельности'' процесса. Используется системный вызов \texttt{waitpid()} с макросом \texttt{WNOHANG}.
\item \texttt{int GetChildPid(ProcessHandle handle);} --- получение pid дочернего процесса.
\item \texttt{SharedMemHandle CreateSharedMemory(const char* name, size\_t size);} --- создание POSIX shared memory объекта. Используется системный вызов \texttt{shm\_open()} и \texttt{ftruncate()}.
\item \texttt{char* MapSharedMemory(SharedMemHandle handle, size\_t size);} --- отображение shared memory в адресное пространство. Используется системный вызов \texttt{mmap()}.
\item \texttt{void UnmapSharedMemory(char* ptr, size\_t size);} --- отмена отображения памяти. Используется системный вызов \texttt{munmap()}.
\item \texttt{void CloseSharedMemory(SharedMemHandle handle);} --- закрытие дескриптора shared memory. Используется системный вызов \texttt{close()}.
\item \texttt{void SendSignal(ProcessHandle handle, int signal);} --- отправка сигнала дочернему процессу. Используется системный вызов \texttt{kill()}.
\item \texttt{void Signal(int sig, SignalHandle handle);} --- установка обработчика сигнала. Используется системный вызов \texttt{signal()}.
\item \texttt{int GetPid();} --- получение pid текущего процесса. Используется системный вызов \texttt{getpid()}.
\item \texttt{int Pause();} --- ожидание сигнала. Используется системный вызов \texttt{pause()}.
\item \texttt{void Sleep(int seconds);} --- ожидание. Используется системный вызов \texttt{sleep()}.
\item \texttt{void Exit(int status);} --- завершение текущего процесса. Используется системный вызов \texttt{\_exit()}.
\end{itemize}

\vspace{2\baselineskip}


\texttt{child.h} --- объявление класса Child. \\
\texttt{src/child.cpp} --- реализация.\\
Поля класса:
\begin{itemize}
\item \texttt{int pid;} --- pid процесса.
\item \texttt{std::string filename;} --- название файла для открытия (создания).
\item \texttt{std::ofstream file;} --- поток для записи в файлы.
\item \texttt{char* shm\_ptr;} --- указатель на отображённую разделяемую память.
\item \texttt{size\_t shm\_size;} --- размер разделяемой памяти.
\item \texttt{std::string shm\_name;} --- имя разделяемой памяти.
Основные функции \texttt{(методы)}:
\item \texttt{void Work();} --- ожидает сигнал, читает строку из разделяемой памяти, инвертирует её и записывает в файл.
\end{itemize}
Основные функции \texttt{(методы)}:
\begin{itemize}
\item \texttt{void Work();} --- ожидает сигнал, читает строку из разделяемой памяти, инвертирует её и записывает в файл.
\end{itemize}

\vspace{2\baselineskip}


\texttt{parent.h} --- объявление класса Parent.\\
\texttt{src/parent.cpp} --- реализация.\\
Поля класса:
\begin{itemize}
\item \texttt{std::random\_device rd;} --- генератор случайных чисел.
\item \texttt{ProcessHandle child1;} --- дескриптор 1 дочернего процесса.
\item \texttt{ProcessHandle child2;} --- дескриптор 2 дочернего процесса.
\item \texttt{SharedMemHandle shm\_handle1;} --- дескриптор разделяемой памяти для 1 дочернего процесса.
\item \texttt{SharedMemHandle shm\_handle2;} --- дескриптор разделяемой памяти для 2 дочернего процесса.
\item \texttt{char* shm\_ptr1;} --- указатель на разделяемую память 1.
\item \texttt{char* shm\_ptr2;} --- указатель на разделяемую память 2.
\item \texttt{std::string shm\_name1;} --- имя разделяемой памяти 1.
\item \texttt{std::string shm\_name2;} --- имя разделяемой памяти 2.
\item \texttt{const size\_t shm\_size;} --- размер разделяемой памяти.
\end{itemize}
Основные функции \texttt{(методы)}:
\begin{itemize}
\item \texttt{void CreateChildProcesses(std::string filename1, std::string filename2);} --- создает 2 дочерних процесса и запускает бинарный файл для дочернего процесса, при ошибке дочерний процесс завершается.
\item \texttt{void Work();} --- получает пользовательский ввод и записывает в одну из областей разделяемой памяти для дочерних процессов по правилу фильтрации, затем отправляет сигнал.
\item \texttt{void EndChildren();} --- завершает дочерние процессы.
\end{itemize} 