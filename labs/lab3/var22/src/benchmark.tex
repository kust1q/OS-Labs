\section{Результаты}

Программа получает на вход названия двух файлов, создает два дочерних процесса, эти процессы в специальной директории (testfiles) открывают (создают) эти файлы.  
Далее все введённые пользователем строки, исключая ``exit'' и ``quit'', инвертируются и записываются в один из файлов в зависимости от правила фильтрации.  
Взаимодействие между родительским и дочерними процессами осуществляется через разделяемую память и сигналы.  
Результатом являются два файла в директории ``testfiles''.  
Если не удалось создать разделяемую память, отобразить её, или дочерние процессы завершились, программа безопасно прекращает свою работу.